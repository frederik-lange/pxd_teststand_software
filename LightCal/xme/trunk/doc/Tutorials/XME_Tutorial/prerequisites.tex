%
% Copyright (c) 2011-2012, fortiss GmbH.
% Licensed under the Apache License, Version 2.0.
% 
% Use, modification and distribution are subject to the terms specified
% in the accompanying license file LICENSE.txt located at the root directory
% of this software distribution. A copy is available at
% http://chromosome.fortiss.org/.
%
% This file is part of CHROMOSOME.
%
% $Id$
%
% Author:
%         Dominik Sojer <sojer@fortiss.org>
%         Michael Geisinger <geisinger@fortiss.org>
%

\section{Prerequisites (20 minutes)}
\xme has been developed with platform independence in mind, but for sake of simplicity,
this tutorial uses a Windows-based development environment and a Windows-based target platform.
%
While installing the proposed build environment, you might want to start reading Section~\ref{sec:architecture} giving an overview on \xme and its features.

Apart from the \xme source archive, two tools are recommended for building \xme applications on Windows.
\begin{itemize}
	\item \textbf{Visual Studio C++} (Express, Professional, Premium or Ultimate, preferably 2008 or 2010 versions):
		Visual Studio is Microsoft's platform for multi-language development.
		The so-called ``Express Edition'' is available free for evaluation purposes\footnote{%
		\url{http://www.microsoft.com/visualstudio/en-us/products/2010-editions/visual-cpp-express}}.
		Note that \xme has not yet been tested with Visual Studio 2011, of which a beta version has been recently released.
		You only have to install the C/C++ compiler, Additional packages like Silverlight or SQL Server are not required.
		Consult Appendix~\ref{appx:install_vs} for details.
	
	\item \textbf{CMake} (at least version 2.8.5):
		CMake is a cross-platform Makefile generator and is used to manage the build system.
		%Input to CMake are a set of \texttt{CMakeLists.txt} files that contain the specifications for the build system.
		Output of CMake are the build system configurations (e.g., a UNIX Makefile, a Microsoft Visual Studio Project, an Eclipse CDT project).
		\xme provides a customized set of macros to deal with components, dependencies, executables and documentation.
		CMake can be downloaded for free\footnote{\url{http://www.cmake.org/cmake/resources/software.html}}.
		\xme currently requires at least CMake version 2.8.5.
		It is \emph{not} necessary to add CMake to the system search path.
		Consult Appendix~\ref{appx:install_cmake} for details.
\end{itemize}
