%
% Copyright (c) 2011-2012, fortiss GmbH.
% Licensed under the Apache License, Version 2.0.
% 
% Use, modification and distribution are subject to the terms specified
% in the accompanying license file LICENSE.txt located at the root directory
% of this software distribution. A copy is available at
% http://chromosome.fortiss.org/.
%
% This file is part of CHROMOSOME.
%
% $Id$
%
% Author:
%         Dominik Sojer <sojer@fortiss.org>
%         Michael Geisinger <geisinger@fortiss.org>
%

\section{Introduction} \label{sec:intro}

For a long time, the focus of embedded systems development has been the implementation of isolated systems with clearly defined borders and interfaces.
Recently, the trend of integrating these complex independent systems into larger \emph{systems of systems} arises:
manufacturing plants get connected with logistics, intelligent cars communicate with each other and the infrastructure. 
Due to the different life cycles of the different involved systems, adaptability in the sense of plug\&play becomes more and more important. Systems must 
be integrated with other systems although the concrete systems were not know at design time. Future embedded systems have to be developed in a way so 
that they can be integrated into these systems of systems without losing their safety, security and real-time capabilities. 

In order to achieve this, a powerful domain-independent software platform is required,
which can flexibly be adapted to various application scenarios.
\xme\footnote{\url{http://chromosome.fortiss.org/}} is a middleware and runtime system intended to meet these requirements. It combines features known
from the embedded domain such as determinism where necessary with adaptivity known from internet technologies. \xme treats extra-functional requirements as 
first-class entity and provides according mechanisms to fulfill the application requirements.

\xme has a large set of designated features and is designed to evolve over time.
It is completely open source and hence transparent to developers and end users.
The current release includes a first subset of the features that will be available in future versions.
Although we have a clear vision of the final system, \xme's development is demand driven.
Its goal is to provide adequate platform support and tooling for maximum efficiency.

The following tutorial will introduce you to the main concepts of \xme and illustrate them with examples that are easy to understand.
Including installation of required prerequisites, this tutorial will take about two hours to complete.

\subsection{The Name \xme}
\xme stands for \textbf{\underbar{Cro}}ss-domain \textbf{\underbar{M}}odular
\textbf{\underbar{O}}perating \textbf{\underbar{S}}ystem \textbf{\underbar{o}}r
\textbf{\underbar{M}}iddlewar\textbf{\underbar{e}}.
The name expresses the vision of \xme:
\begin{enumerate}
	\item We believe that in the future cross-domain solutions will be required.
	\item As different applications might have many different requirements
		and the middleware will be deployed to very heterogeneous platforms,
		a scalable and modular solution is required.
	\item \xme will sometimes operate on top of an operating system (OS),
		but it might also replace an OS due to resource reasons.
		Therefore, we think that in the future,
		the boundary between an operating system and a middleware might get blurred.
\end{enumerate}

Similar to biology, a \xme instance can be built up a number of genes (modules).
It is not the intention of \xme to invent new genes, but instead to use
existing protocols / solutions to implement components for specific tasks.
The major idea is that \xme offers a flexible blueprint that enables the
selection of different solutions to adapt a system to the specific requirements.

Analog to the evolution of mankind, we do not believe that the genes,
but also the blueprint provided by \xme are initialy perfect.
Instead we believe in constant improvement based on discussions about specific components.
This is one of the reasons why we are offering \xme as open-source software.
In case you have any questions or suggestions for improvement, we are looking forward to your comments.

\subsection{Questions and Contact Information}

If you have any questions with respect to this tutorial or want to report a bug,
please check the Frequently Asked Questions in Appendix~\ref{appx:faq}.
If your question is not answered, please do not hesitate to send an e-mail to \url{chromosome@fortiss.org}. Thank you!

\subsection{Acknowledgements}

This work is partially funded by the following research grants:
\begin{itemize}
	\item German Federal Ministry of Economics and Technology (BMWi) research grant 01ME12009
		(project RACE\footnote{\url{http://www.projekt-race.de/}})
	\item German Federal Ministry of Economics and Technology (BMWi) research grant 01MA11002
		(project AutoPnP\footnote{\url{http://www.autopnp.com/}})
	\item Bavarian Ministry of Economic Affairs, Infrastructure, Transport and Technology grant programme 1330/686
		(Vorlaufforschung Fortiss)
\end{itemize}